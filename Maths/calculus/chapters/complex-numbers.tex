\chapter{Complex Numbers}

\section{Imaginary Numbers}
\begin{equation*}
    i = \sqrt{-1}
\end{equation*}

\begin{equation*}
    z = x + iy
\end{equation*}

\begin{equation*}
    z_1 + z_2 = (x_1 + iy_1) + (x_2 + iy_2) = (x_1 + x_2) + i(y_1 + y_2)
\end{equation*}

\begin{align*}
    z_1 + z_2 &= z_2 + z_1\\
    z_1 + (z_2 + z_3) &= (z_1 + z_2) + z_3
\end{align*}

\begin{align*}
    z_1z_2 &= (x_1 + iy_1)(x_2 + iy_2)\\
    &= x_1x_2 + i(x_1y_2 + y_1x_2) + i^2 y_1y_2\\
    &= (x_1x_2 - y_1y_2) +i(x_1y_2 - y_1x_2)
\end{align*}

\begin{align*}
    z_1z_2 &= z_2z_1\\
    (z_1z_2)z_3 &= z_1(z_2z_3)
\end{align*}

\begin{align*}
    \frac{z_1}{z_2} = \frac{x_1 + iy_1}{x_2 + iy_2}
\end{align*}

\begin{equation*}
    |z| = \sqrt{x^2 + y^2}
\end{equation*}

\begin{equation*}
    \arg(z) = \arctan\left(\frac{y}{x}\right)
\end{equation*}

\section{Polar Representation}

\begin{equation*}
    e^z = 1 + z + \frac{z^2}{2!} + \frac{z^3}{3!} + \dots
\end{equation*}

\begin{align*}
    e^{i\theta} &= 1 + i\theta + \frac{(i\theta)^{2}}{2!} + \frac{(i\theta)^{3}}{3!} + \frac{(i\theta)^{4}}{4!}\\
    &= 1 + i\theta - \frac{\theta^{2}}{2!} - \frac{i\theta^{3}}{3!} + \frac{\theta^{4}}{4!} + \dots\\
    &= \left(1 - \frac{\theta^{2}}{2!} + \frac{\theta^{4}}{4!} + \dots\right) + i\left(\theta - \frac{i\theta^{3}}{3!} + \frac{i\theta^{5}}{5!} + \dots\right)
\end{align*}

\begin{equation*}
    e^{i\theta} = \cos\theta + i\sin\theta
\end{equation*}

\begin{equation*}
    e^{in\theta} = \cos n\theta + i \sin n\theta
\end{equation*}

\begin{equation*}
    z = re^{i\theta}
\end{equation*}

\begin{equation*}
    re^{i\theta} = re^{i(\theta + 2n\pi)}
\end{equation*}

\section{De Moivre's Theorem}
\begin{equation*}
    (\cos\theta + i\sin\theta)^n = \cos(n\theta) + i\sin(n\theta)
\end{equation*}

\begin{equation*}
    z + \frac{1}{z} = 2\cos\theta
\end{equation*}

\begin{equation*}
    z - \frac{1}{z} = 2i\sin\theta
\end{equation*}

\section{Hyperbolic Functions}

\begin{equation*}
    \cosh(x) = \frac{1}{2}(e^x + e^{-x})
\end{equation*}

\begin{equation*}
    \sinh(x) = \frac{1}{2}(e^x - e^{-x})
\end{equation*}

\begin{equation*}
    \tanh(x) = \frac{\sinh(x)}{\cosh(x)} = \frac{e^x - e^{-x}}{e^x + e^{-x}}
\end{equation*}

\begin{align*}
    \cos(ix) &= \frac{1}{2}(e^x + e^{-x})\\
    \sin(ix) &= \frac{1}{2}(e^x - e^{-x})\\
\end{align*}

\begin{align*}
    \cosh(x) &= \cos(ix)\\
    i\sinh(x) &= \sin(ix)\\
    \cos(x) &= \cosh(ix)\\
    i\sin(x) &= \sinh(ix)
\end{align*}

\begin{equation*}
    \frac{d}{dx}\cosh(x) = \sinh(x)
\end{equation*}

\begin{equation*}
    \frac{d}{dx}\sinh(x) = \cosh(x)
\end{equation*}