\chapter{Differentiation}
\section{The Derivative}

The derivative measures change. More specifically, if we have a function, f(x), then the derivative
of this function (often denoted f'(x)) tells us how quickly the function is increasing or decreasing
at any given point. If we consider a general function and think how we might measure how it changes,
we may look at the difference in the function between two values of x.

\begin{center}
    \begin{tikzpicture}[scale=1]
    \begin{axis}[
        axis lines = left,
        axis lines = center,
        xtick style={draw=none},
        ytick style={draw=none},
        xticklabels={},
        yticklabels={},
        xlabel = {x},
        ylabel = {f(x)},
    ]
    
    \addplot [
        domain=-0:12, 
        samples=100, 
        color=black,
    ]
    {x^2};
    \addplot+ [
        black, dotted,
        mark=none,
    ]
    coordinates {
        (8,0)
        (8,64)

        (10,0)
        (10,100)

        (0, 64)
        (8, 64)

        (0, 100)
        (10, 100)

    };
     
    \end{axis}
    \end{tikzpicture}
\end{center}

Here we have two points, one at $x = x_1$ and $x = x_2 = x_1 + \Delta x$. Here $\Delta$ means "change in"
and so $\Delta x$ denotes the difference between our two x values. The corresponding function values are
$f(x_1)$ and $f(x_1 + \Delta x)$ where the difference between these two values is $\Delta f$. We can say 
a reasonable estimate of the rate of change would be the gradient of the line that passes through these two 
points. We can write this as

\begin{equation*}
    \frac{\Delta f}{\Delta x} = \frac{f(x + \Delta x) - f(x)}{\Delta x}
\end{equation*}

We can see that this estimate for the rate of change gets better the closer the two points are, or the smaller 
$\Delta x$ gets. If we take the limit of $\Delta x \to 0$, then our two points become infinitesimally close and the
line through them becomes the tangent line to the curve. Thus we arrive at the definition of the derivative and the
idea of "instantaneous" rate of change. 

\begin{equation*}
    \frac{df}{dx} = \lim_{\Delta x \to 0} \frac{\Delta f}{\Delta x}
\end{equation*}

\begin{equation*}
    \frac{df}{dx} = \lim_{\Delta x \to 0} \frac{f(x + \Delta x) - f(x)}{\Delta x}
\end{equation*}

\begin{align*}
    \frac{d}{dx}(x^{n}) &= \lim_{\Delta x \to 0} \frac{(x + \Delta x)^{n} - x^{n}}{\Delta x}\\
    &= \lim_{\Delta x \to 0} \frac{(x^{n} + n x^{n-1}\Delta x + \frac{1}{2}n(n-1)x^{n-2}\Delta x^{2} +\dots) - x^{n}}{\Delta x}\\
    &= \lim_{\Delta x \to 0} \frac{n x^{n-1}\Delta x + \frac{1}{2}n(n-1)x^{n-2}\Delta x^{2}}{\Delta x}\\
    &= \lim_{\Delta x \to 0} n x^{n-1} + \frac{1}{2}n(n-1)x^{n-2}\Delta x\\
    &= n x^{n-1}
\end{align*}

\begin{equation*}
    \frac{d}{dx}(e^{ax}) = ae^{ax}
\end{equation*}

\section{The Chain Rule}

\begin{equation*}
    \frac{d}{dx}f[g(x)] = \frac{df}{dg} \frac{dg}{dx}
\end{equation*}

\section{The Product Rule}

\begin{equation*}
    \frac{d}{dx} \left[f(x)g(x)\right] = \frac{df}{dx} g(x) + f(x)\frac{dg}{dx}
\end{equation*}

\begin{equation*}
    (uv)' = u' v + uv'
\end{equation*}

\begin{equation*}
    \frac{d}{dx} \left[f(x)g(x)h(x)\right] = \frac{df}{dx} g(x)h(x) + f(x)\frac{dg}{dx} h(x) + f(x)g(x) \frac{dh}{dx}
\end{equation*}

\begin{align*}
    \frac{d}{dx}\left(\frac{f(x)}{g(x)}\right) &= \frac{df}{dx}\left(\frac{1}{g(x)}\right) + f(x)\frac{d}{dx}\left(\frac{1}{g(x)}\right)\\
    &= \frac{f'(x)}{g(x)} + f(x)\left(- \frac{g'(x)}{(g(x))^2}\right)\\
    &= \frac{f'(x)}{g(x)} - \frac{f(x)g'(x)}{(g(x))^{2}}\\
    &= \frac{f'(x)g(x) - f(x)g'(x)}{(g(x))^{2}}
\end{align*}

\section{Implicit Differentiation}