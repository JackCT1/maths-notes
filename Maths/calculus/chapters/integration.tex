\chapter{Integration}
\section{The Integral}

In essence, an integral is just a sum. A sum of infinitely many infitetesimally small parts.
Integrals are most commonly visualised as the area under a curve. If wanted to estimate the
area under a curve we might seek to divide the area into vertical rectangles, each with a 
thickness $\Delta x$ and a height of $f(x)$, and sum the combined area of all of these rectangles. 
We can see that this estimate of the area under the curve gets more accurate when we use 
thinner rectangles to the point where if we take the limit $\Delta x \to 0$, we have the 
exact area. We write this notationally as 

\begin{equation*}
    I = \int_{a}^{b} f(x)dx
\end{equation*}

\[\int_{a}^{b}0dx = 0, \qquad \int_{a}^{a}f(x)dx\]

\begin{equation*}
    \int_{a}^{c}f(x)dx = \int_{a}^{b}f(x)dx + \int_{b}^{c}f(x)dx
\end{equation*}

\begin{equation*}
    \int_{a}^{b}f(x)dx = - \int_{b}^{a}f(x)dx
\end{equation*}

A key idea to note is that integration and differentiation are inherently linked. In fact,
they are the inverse processes of each other. In other words, differentiating a function and 
then integrating it will return the exact same function, no change. The integral is the 
"antiderivative" so to speak To prove this, suppose we have a function $F(x)$ which is the 
integral of another function $f(u)$ between the points $a$ and $x$.

\begin{equation*}
    F(x) = \int_{a}^{x}f(u)du
\end{equation*}

$a$ is fixed and so $x$ is the only variable. Now if we consider the value of $F$ at $x + \Delta x$
then we get

\begin{align*}
    F(x + \Delta x) &= \int_{a}^{x + \Delta x}f(u)du\\
    &= \int_{a}^{x}f(u)du + \int_{x}^{x + \Delta x}f(u)du\\
    &= F(x) + \int_{x}^{x + \Delta x}f(u)du\\
\end{align*}

Now dividing both sides by $\Delta x$,

\begin{equation*}
    \frac{F(x + \Delta x) - F(x)}{\Delta x} = 
    \frac{1}{\Delta x}\int_{x}^{x + \Delta x}f(u)du
\end{equation*}

We can see that the left hand side is just the derivative of $F(x)$ and the integral on the 
right evaluates to $f(x)\Delta x$ if $\Delta x $ is sufficiently small. This now gives us

\begin{equation*}
    \frac{dF(x)}{dx} = \frac{1}{\Delta x}f(x)\Delta x = f(x)
\end{equation*}

Using our initial expression for $F(x)$, we can rewrite this as

\begin{equation*}
    \frac{d}{dx}\left[\int_{a}^{x}f(u)du\right] = f(x)
\end{equation*}

Thus showing the inverse relationship between derivatives and integrals. This relation is called
\textbf{The Fundamental Theorem of Calculus} and is the basis of everything we will do going 
forward.

\begin{equation*}
    \int x^{n}dx = \frac{x^{n+1}}{n+1} + C
\end{equation*}

\[\int e^{ax}dx = \frac{1}{a}e^{ax} + c, \qquad \int \frac{1}{x}dx = ln(x) + c\]

\section{Integration by Substitution}
An integral can often be made simpler when we make a change of variables, also called substitution.
Deciding what substitution is best comes from practice and can be a bit of an art form. The following
are some useful examples.

\section{Integration by Parts}

We can derive a useful relation for integration using the product rule.

\begin{equation*}
    \frac{d}{dx}(uv) = u \frac{dv}{dx} + \frac{du}{dx} v
\end{equation*}

\begin{equation*}
    uv = \int u \frac{dv}{dx} dx + \int \frac{du}{dx}vdx
\end{equation*}
    
\begin{equation*}
    \int u \frac{dv}{dx} dx = uv - \int \frac{du}{dx}vdx
\end{equation*}

\section{Reduction Formulae}

\section{Surfaces and Volumes of Revolution}

\begin{equation*}
    \Delta s \approx \sqrt{(\Delta x)^{2} + (\Delta y)^{2}}
\end{equation*}

\begin{equation*}
    \frac{ds}{dx} = \sqrt{1 + \left(\frac{dy}{dx}\right)^2}
\end{equation*}

\begin{equation*}
    s = \int_{a}^{b}\sqrt{1 + \left(\frac{dy}{dx}\right)^{2}}dx
\end{equation*}

\begin{equation*}
    S = \int_{a}^{b} 2 \pi y ds
\end{equation*}

\begin{equation*}
    S = \int_{a}^{b} 2 \pi y \sqrt{1 + \left(\frac{dy}{dx}\right)^{2}}dx
\end{equation*}

\begin{equation*}
    V = \int_{a}^{b} \pi y^{2} dx
\end{equation*}

\section{Plane Polar Coordinates}

\begin{center}
    \begin{tikzpicture}
        \draw [<->] (3,0) -- (0,0) -- (0,3);
        \draw (-3,0) -- (0,0) -- (0,-3);
        \draw (0,0) circle [radius = 2];
        \draw [->] (0,0) -- (1.41, 1.41);
        \draw (0.5,0) arc [radius = 0.5, start angle = 0, end angle = 45];
        \node at (0.6,0.25) {\footnotesize{\(\theta\)}};
        \node [above left] at (0.7,0.7) {\(r\)};
        \node [above right] at (1.41,1.41) {\(P\)};
        \node [right] at (3,0) {\(x\)};
        \node [above] at (0,3) {\(y\)};
    \end{tikzpicture}
\end{center}

\begin{equation*}
    A = \int_{\phi_1}^{\phi_2}\frac{1}{2}\rho^2 d\phi
\end{equation*}