\documentclass[]{article}

\usepackage{circuitikz}
\usepackage{amsmath}
\usepackage{tikz}
\usepackage{tikzsymbols}
\usepackage{pgfplots}
\usepackage{layout}
\usepackage[left=25mm, right=25mm]{geometry}

\pgfplotsset{compat=1.18}
\begin{document}

\title{Calculus Notes}
\author{Jack Thomas}
\date{}

\maketitle
\newpage

\tableofcontents
\newpage

\part{Single Variable Calculus}
\section{Differentiation}
\subsection{The Derivative}

\begin{center}
    \begin{tikzpicture}[scale=0.7]
    \begin{axis}[
        axis lines = left,
        axis lines = center,
        xtick style={draw=none},
        ytick style={draw=none},
        xticklabels={},
        yticklabels={},
        xlabel = {x},
        ylabel = {y},
    ]
    
    \addplot [
        domain=-15.708:15.708, 
        samples=100, 
        color=black,
    ]
    {x};
     
    \end{axis}
    \end{tikzpicture}
\end{center}

\begin{center}
    \begin{tikzpicture}[scale=1]
    \begin{axis}[
        axis lines = left,
        axis lines = center,
        xtick style={draw=none},
        ytick style={draw=none},
        xticklabels={},
        yticklabels={},
        xlabel = {x},
        ylabel = {y},
    ]
    
    \addplot [
        domain=-15.708:15.708, 
        samples=100, 
        color=black,
    ]
    {x^2};
     
    \end{axis}
    \end{tikzpicture}
\end{center}

\begin{equation*}
    \frac{df}{dx} = \lim_{\Delta x \to 0} \frac{\Delta f}{\Delta x}
\end{equation*}

\begin{equation*}
    \frac{df}{dx} = \lim_{\Delta x \to 0} \frac{f(x + \Delta x) - f(x)}{\Delta x}
\end{equation*}

\begin{align*}
    \frac{d}{dx}(x^{n}) &= \lim_{\Delta x \to 0} \frac{(x + \Delta x)^{n} - x^{n}}{\Delta x}\\
    &= \lim_{\Delta x \to 0} \frac{(x^{n} + n x^{n-1}\Delta x + \frac{1}{2}n(n-1)x^{n-2}\Delta x^{2} +\dots) - x^{n}}{\Delta x}\\
    &= \lim_{\Delta x \to 0} \frac{n x^{n-1}\Delta x + \frac{1}{2}n(n-1)x^{n-2}\Delta x^{2}}{\Delta x}\\
    &= \lim_{\Delta x \to 0} n x^{n-1} + \frac{1}{2}n(n-1)x^{n-2}\Delta x\\
    &= n x^{n-1}
\end{align*}

\begin{equation*}
    \frac{d}{dx}(e^{ax}) = ae^{ax}
\end{equation*}

\subsection{The Chain Rule}

\begin{equation*}
    \frac{d}{dx}f[g(x)] = \frac{df}{dg} \frac{dg}{dx}
\end{equation*}

\subsection{The Product Rule}

\begin{equation*}
    \frac{d}{dx} f(x)g(x) = \frac{df}{dx} g(x) + f(x)\frac{dg}{dx}
\end{equation*}

\begin{equation*}
    \frac{d}{dx} f(x)g(x)h(x) = \frac{df}{dx} g(x)h(x) + f(x)\frac{dg}{dx} h(x) + f(x)g(x) \frac{dh}{dx}
\end{equation*}

\begin{align*}
    \frac{d}{dx}\left(\frac{f(x)}{g(x)}\right) &= \frac{df}{dx}\left(\frac{1}{g(x)}\right) + f(x)\frac{d}{dx}\left(\frac{1}{g(x)}\right)\\
    &= \frac{f'(x)}{g(x)} + f(x)\left(- \frac{g'(x)}{(g(x))^2}\right)\\
    &= \frac{f'(x)}{g(x)} - \frac{f(x)g'(x)}{(g(x))^{2}}\\
    &= \frac{f'(x)g(x) - f(x)g'(x)}{(g(x))^{2}}
\end{align*}

\section{Integration}
\subsection{The Antiderivative}

\begin{equation*}
    I = \int_{a}^{b} f(x)dx
\end{equation*}

\begin{equation*}
    \int x^{n}dx = \frac{x^{n+1}}{n+1} + C
\end{equation*}

\[\int e^{ax}dx = \frac{1}{a}e^{ax} + c, \space \int \frac{1}{x}dx = ln(x) + c\]

\subsection{Integration by Substitution}
\subsection{Integration by Parts}

\begin{equation*}
    \frac{d}{dx}(uv) = u \frac{dv}{dx} + \frac{du}{dx} v
\end{equation*}

\begin{equation*}
    uv = \int u \frac{dv}{dx} dx + \int \frac{du}{dx}vdx
\end{equation*}
    
\begin{equation*}
    \int u \frac{dv}{dx} dx = uv - \int \frac{du}{dx}vdx
\end{equation*}

\subsection{Surfaces and Volumes of Revolution}

\begin{equation*}
    S = \int_{a}^{b} 2 \pi y ds
\end{equation*}

\begin{equation*}
    S = \int_{a}^{b} 2 \pi y \sqrt{1 + \left(\frac{dy}{dx}\right)^{2}}dx
\end{equation*}

\[ 1 \times  2 = 3,\qquad 2\times 3 = 1, \qquad 3\times 1 = 2\]

\part{Multi Variable Calculus}
\section{Partial Differentiation}
\section{Multiple Integrals}
\section{Vector Calculus}

\end{document}